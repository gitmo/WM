%
%  DBS-Project
%
%  NvG, FH, SR
%
\documentclass[11pt,german]{scrartcl}

% See geometry.pdf to learn the layout options.
\usepackage{geometry}
\geometry{a4paper}

% To begin paragraphs with an empty line
\usepackage[parfill]{parskip}

% Use utf-8 encoding for foreign characters
\usepackage[utf8]{inputenc}
\usepackage[T1]{fontenc}

% Setup for fullpage use
\usepackage{fullpage}

% More symbols
\usepackage{amsmath}
\usepackage{amssymb}
%\usepackage{latexsym}

% Package for including code in the document
\usepackage{listings}
\usepackage{color}
\usepackage{moreverb}


\title{Anwendungsfälle}
\author{NvG, FH, SR}

% Activate to display a given date or no date
\date{}

%%%%%%%%%%%%%%%%%%%%%%%%%%%%%%%%%%%%%%%%%%%%%%%%%%%%%%%%%%%%%%%%%%
\begin{document}

\renewcommand{\labelenumi}{\alph{enumi})}
\renewcommand{\labelenumii}{$\bullet$}

% define a macro ausgabe which takes as argument
% your ausgabe.txt file’s name 
% \newcommand{\ausgabe}[1] {\hrule\small\verbatiminput{#1}\normalsize\hrule}

\definecolor{light-gray}{gray}{0.95}

\lstset{
  language=Java,               % choose the language of the code
  basicstyle=\scriptsize,   % size of fonts used for the code
  numbers=left,             % where to put the line-numbers
  numberstyle=\tiny,        % size of fonts for used line-numbers
  stepnumber=1,             % step between line-numbers
  numbersep=5pt,            % how far the line-numbers are from the code
  backgroundcolor=\color{light-gray},  % background color, add \usepackage{color}
  showspaces=false,         % show spaces adding particular underscores
  showstringspaces=false,   % underline spaces within strings
  showtabs=false,           % show tabs within strings adding particular underscores
  frame=single,             % adds a frame around the code
  tabsize=4,                % sets default tabsize to 2 spaces
  captionpos=b,             % sets the caption-position to bottom
  breaklines=true,          % sets automatic line breaking
  breakatwhitespace=false,  % sets if automatic breaks should only happen at whitespace
  escapeinside={\%*}{*)},   % if you want to add a comment within your code
  extendedchars=true,
  literate=%
    {Ö}{{\"O}}1
    {Ä}{{\"A}}1
    {Ü}{{\"U}}1
    {ß}{{\ss}}2
    {ü}{{\"u}}1
    {ä}{{\"a}}1
    {ö}{{\"o}}1
    {°}{{}}0
}


\maketitle

%%%%%%%%%%%%%%%%%%%%%%%%%%%%%%%%%%%%%%%%%%%%%%%%%%%%%%%%%%%%%%%%%%

%\setcounter{section}{5}

%\clearpage

\section{Einsatz eintragen}
Die methode {\it insertPlayerToMatch(Player player, Match match);} ist in {\it MatchService} implementiert. Falls aller Voraussetzungen erfüllt sind wird ein {\it LineUpEvent} generiert und dem Spiel hinzugefügt. Danach wird das Match gespeichert um die Änderungen konsistent zu machen.\\
Nötige Voraussetzungen für das erstellen eines {\it LineUpEvents} sind:\\
\begin{itemize}
	\item Der Spieler gehört zu einem der spielenden Teams.
	\item Das Team für das der Spieler spielt hat noch keine 11 Spieler in der Startaufstellung.
\end{itemize}
Um Spieler in das Spiel einzügen, nachdem das Spiel begonnen hat existiert die {\it substitutePlayers(Player playerOut, PlayerIn, Integer Minute, Match match);}. Hierbei wird der Spieler {\it PlayerOut} durch {\it PlayerIn} ersetzt, falls {\it PlayerOut} zur gegebenen {\it Minute} im Spiel({\it match}) ist, beide in der selben Manschaft spielen, die Mannschaft am Spiel teilnimmt und {\it PlayerIn} davor noch nicht gespielt hat.\\
Zum Austauschen der Spieler wird ein {\it SubstitutionEvent} erstellt und dem {\it Match} hinzugefügt.

\section{Tor eintragen}
Die Methode {\it insertGoal(GoalEvent goal, Player player, Match match);} ist in der{\it MatchService}-Klasse implementiert. Beim {\it goal} müssen das Team, dass das Tor gutgeschrieben bekommt und die Spielminute eingetragen sein. Falls der Spieler im Match spielt, wird das Tor hinzugefügt.\\
Ein Eigentor wird dadurch erstellt, dass im {\it GoalEvent} die Mannschaft des Gegners angegeben wird.

\section{Ergebnis ausgeben}
Die Methode {\it getResult(Match match);} ist auch in {\it MatchService} implementiert. Es wird entweder das Zwischen- oder das Endergebnis ausgegeben, je nachdem, wann die Methode aufgerufen wird.

\section{Gruppentabelle berechnen}
Die Methode {\it printStandings(TournamentGroup group);} ist in {\it GroupService} implementiert. Falls die Spiele in der Gruppe schon initialisiert wurden wird die Tabelle ausgegeben. Die Tabelle besteht 4 Zeilen , die aus Rang, Teamname, gespielten Spielen, Torverhältnis und Punkten bestehen. 

\section{Sieger ausgeben}
Die Methode {\it weAreTheChampions(Tournament tournament);} ist in {\it TournamentService} implementiert. Es werden die beiden Finalisten und der Dritte des Turniers ausgegeben.

\section{aufrufende Funktionen}
Alle Funktionen werden in den Testklassen ihres Service-Layers getestet. Die Testklassen befinden sich im Packet {\it dbs.project.service} im Ordner {\it src/test/} und heißen $\langle$ {\it KlasseServiceTest} $\rangle$ .java, wobei Klasse natürlich für die jeweilige Klasse steht.

%\input{}

\end{document}
