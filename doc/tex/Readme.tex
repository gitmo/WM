%
%  DBS-Project
%
%  NvG, FH, SR
%
\documentclass[11pt,german]{scrartcl}

% See geometry.pdf to learn the layout options.
\usepackage{geometry}
\geometry{a4paper}

% To begin paragraphs with an empty line
\usepackage[parfill]{parskip}

% Use utf-8 encoding for foreign characters
\usepackage[utf8]{inputenc}
\usepackage[T1]{fontenc}

% Setup for fullpage use
\usepackage{fullpage}

% More symbols
\usepackage{amsmath}
\usepackage{amssymb}
%\usepackage{latexsym}

% Package for including code in the document
\usepackage{listings}
\usepackage{color}
\usepackage{moreverb}

% This is now the recommended way for checking for PDFLaTeX:
\usepackage{ifpdf}

%\newif\ifpdf
%\ifx\pdfoutput\undefined
%\pdffalse % we are not running PDFLaTeX
%\else
%\pdfoutput=1 % we are running PDFLaTeX
%\pdftrue
%\fi

\ifpdf
\usepackage[pdftex]{graphicx}
\else
\usepackage{graphicx}
\fi

\lstset{
  language=BASH,             % choose the language of the code
  basicstyle=\scriptsize,   % size of fonts used for the code
  numbers=left,             % where to put the line-numbers
  numberstyle=\tiny,        % size of fonts for used line-numbers
  stepnumber=1,             % step between line-numbers
  numbersep=5pt,            % how far the line-numbers are from the code
  backgroundcolor=\color{light-gray},  % background color, add \usepackage{color}
  showspaces=false,         % show spaces adding particular underscores
  showstringspaces=false,   % underline spaces within strings
  showtabs=false,           % show tabs within strings adding particular underscores
  frame=single,             % adds a frame around the code
  tabsize=4,                % sets default tabsize to 2 spaces
  captionpos=b,             % sets the caption-position to bottom
  breaklines=true,          % sets automatic line breaking
  breakatwhitespace=false,  % sets if automatic breaks should only happen at whitespace
  escapeinside={\%*}{*)},   % if you want to add a comment within your code
  extendedchars=true,
  literate=%
    {Ö}{{\"O}}1
    {Ä}{{\"A}}1
    {Ü}{{\"U}}1
    {ß}{{\ss}}2
    {ü}{{\"u}}1
    {ä}{{\"a}}1
    {ö}{{\"o}}1
    {°}{{}}0
}


\title{Readme}
\author{Nico von Geyso, Felix Höffken, Sebastian Raitza}

% Activate to display a given date or no date
\date{}

%%%%%%%%%%%%%%%%%%%%%%%%%%%%%%%%%%%%%%%%%%%%%%%%%%%%%%%%%%%%%%%%%%
\begin{document}

\renewcommand{\labelenumi}{\alph{enumi})}
\renewcommand{\labelenumii}{$\bullet$}

% define a macro ausgabe which takes as argument
% your ausgabe.txt file’s name 
% \newcommand{\ausgabe}[1] {\hrule\small\verbatiminput{#1}\normalsize\hrule}

\definecolor{light-gray}{gray}{0.95}

\lstset{
  language=SQL,             % choose the language of the code
  basicstyle=\ttfamily\small, 
  numbers=left,             % where to put the line-numbers
  numberstyle=\tiny,        % size of fonts for used line-numbers
  stepnumber=0,             % step between line-numbers
  numbersep=5pt,            % how far the line-numbers are from the code
  backgroundcolor=\color{light-gray},  % background color, add \usepackage{color}
  showspaces=false,         % show spaces adding particular underscores
  showstringspaces=false,   % underline spaces within strings
  showtabs=false,           % show tabs within strings adding particular underscores
  frame=single,             % adds a frame around the code
  tabsize=4,                % sets default tabsize to 2 spaces
  captionpos=b,             % sets the caption-position to bottom
  breaklines=true,          % sets automatic line breaking
  breakatwhitespace=false,  % sets if automatic breaks should only happen at whitespace
  escapeinside={\%*}{*)},   % if you want to add a comment within your code
  extendedchars=true,
  literate=%
    {Ö}{{\"O}}1
    {Ä}{{\"A}}1
    {Ü}{{\"U}}1
    {ß}{{\ss}}2
    {ü}{{\"u}}1
    {ä}{{\"a}}1
    {ö}{{\"o}}1
    {°}{{}}0
}


\maketitle

%%%%%%%%%%%%%%%%%%%%%%%%%%%%%%%%%%%%%%%%%%%%%%%%%%%%%%%%%%%%%%%%%%

\section*{Installationsanleitung}
\subsection*{Programm}
\begin{enumerate}
\item Starte den PostgreSql-Server
\item Erstelle folgenden Postgres-Benutzer
\begin{verbatim}Datenbankname: test
Benutzername: postgres
Benutzerpassword: postgres\end{verbatim}

\item Erstelle das Schema (mit Tabellen und Funktionen) aus der Datei {\tt create.sql}
\begin{lstlisting}
psql --username=postgres -h localhost test < doc/sql/create.sql
\end{lstlisting}

\item Die GUI starten mit:
\begin{lstlisting}
java -Dfile.encoding=UTF-8 -jar WM-0.0.1-SNAPSHOT-jar-with-dependencies.jar
\end{lstlisting}
\end{enumerate}



\subsection*{Tests}

\begin{enumerate}
\item Starten der Tests (mit Maven)
\begin{lstlisting}
./runtests.sh
\end{lstlisting}
Dies löscht das Schema aus der Datenbank, extrahiert den Source-Tree und führt {\tt mvn test} aus. Das Schema wird dabei automatisch im \emph{Maven buildcyle} mittels \emph{Hibernate} erstellt und in die Datenbank exportiert.
\end{enumerate}


\section*{Gut zu wissen...}
\begin{itemize}
\item Es kann passieren, dass nach einer Turniererstellung (stored procedure oder java generator) keine Veränderung sichtbar werden.
Dies liegt meistens daran, dass bereits ein Turnier mit der Jahreszahl in der Datenbank eingetragen ist.

\item Die Turnier-Simulation ist extrem zeitaufwendig, da für jedes Spiel der Gruppenphase 22 Spieler per seperaten einzelnen Anfragen aufgestellt werden und dann nochmals 10 zufällige MatchEvents generiert werden.

\item Das vermischte Generieren der Turniere (Java-Generator und Stored procedure) kann zu Problemen führen, da anscheinend Hibernate anders als Postgres {\it sequences} verwaltet.
\end{itemize}

\end{document}
