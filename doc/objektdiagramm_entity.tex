%
%  DBS-Project
%
%  NvG, FH, SR
%
\documentclass[11pt,german]{scrartcl}

% See geometry.pdf to learn the layout options.
\usepackage{geometry}
\geometry{a4paper}

% To begin paragraphs with an empty line
\usepackage[parfill]{parskip}

% Use utf-8 encoding for foreign characters
\usepackage[utf8]{inputenc}
\usepackage[T1]{fontenc}

% Setup for fullpage use
\usepackage{fullpage}

% More symbols
\usepackage{amsmath}
\usepackage{amssymb}
%\usepackage{latexsym}

% Package for including code in the document
\usepackage{listings}
\usepackage{color}
\usepackage{moreverb}


\title{Objektmodell: Design}
\author{NvG, FH, SR}

% Activate to display a given date or no date
\date{}

%%%%%%%%%%%%%%%%%%%%%%%%%%%%%%%%%%%%%%%%%%%%%%%%%%%%%%%%%%%%%%%%%%
\begin{document}

\renewcommand{\labelenumi}{\alph{enumi})}
\renewcommand{\labelenumii}{$\bullet$}

% define a macro ausgabe which takes as argument
% your ausgabe.txt file’s name 
% \newcommand{\ausgabe}[1] {\hrule\small\verbatiminput{#1}\normalsize\hrule}

\definecolor{light-gray}{gray}{0.95}

\lstset{
  language=Java,               % choose the language of the code
  basicstyle=\scriptsize,   % size of fonts used for the code
  numbers=left,             % where to put the line-numbers
  numberstyle=\tiny,        % size of fonts for used line-numbers
  stepnumber=1,             % step between line-numbers
  numbersep=5pt,            % how far the line-numbers are from the code
  backgroundcolor=\color{light-gray},  % background color, add \usepackage{color}
  showspaces=false,         % show spaces adding particular underscores
  showstringspaces=false,   % underline spaces within strings
  showtabs=false,           % show tabs within strings adding particular underscores
  frame=single,             % adds a frame around the code
  tabsize=4,                % sets default tabsize to 2 spaces
  captionpos=b,             % sets the caption-position to bottom
  breaklines=true,          % sets automatic line breaking
  breakatwhitespace=false,  % sets if automatic breaks should only happen at whitespace
  escapeinside={\%*}{*)},   % if you want to add a comment within your code
  extendedchars=true,
  literate=%
    {Ö}{{\"O}}1
    {Ä}{{\"A}}1
    {Ü}{{\"U}}1
    {ß}{{\ss}}2
    {ü}{{\"u}}1
    {ä}{{\"a}}1
    {ö}{{\"o}}1
    {°}{{}}0
}


\maketitle

%%%%%%%%%%%%%%%%%%%%%%%%%%%%%%%%%%%%%%%%%%%%%%%%%%%%%%%%%%%%%%%%%%

%\setcounter{section}{5}

%\clearpage
\section{Packet: dbs.project.entity}

\subsection{Advisor}
Ein Advisor repräsentiert eine Person, die für ein Team arbeitet, z.B. Trainer, TeamArzt, ...

\subsection{Country}
Country repräsentiert ein Land.

\subsection{GroupMatch}
Ein GroupMatch ist ein Spiel, dass in der Gruppenphase ausgetragen wird.

\subsection{GroupStage}
Die GroupStage ist die Gruppenphase des Turniers.

\subsection{KnockOutMatch}
Ein KnockOutMatch ist ein Spiel, dass in der KO-Runde ausgetragen wird.

\subsection{Match}
Ein Match ist ein Spiel, also die Oberklasse von KnockOutMatch und GroupMatch. Sie ist {\it abstact} kann also nicht implementiert werden.

\subsection{MatchEvent}
Ein MatchEvent ist ein Event, dass während eines Spiels auftreten kann. Es ist {\it abstact} kann also nicht direkt implementiert werden.
Es gibt mehrere Unterklassen, die später erklärt werden.

\subsection{Person}
Eine Person ist eine Person, die in dem Turnier auftritt. Sie ist {\it abstract} und die Überklasse von Player und Advisor.

\subsection{Player}
Ein Player repräsentiert einen Spieler im Turnier.

\subsection{Stadium}
Ein Stadium repräsentiert ein Stadion.

\subsection{Team}
Ein Team ist eine Gruppe aus Advisorn und Spielern, usw. Sie repräsentieren die Mannschafte, die gegeneinander spielen.

\subsection{Tournament}
Ein Tournament ist ein Turnier - da mehrere Turniere in unsere Datenbank passen sollen ist sie nötig.

\subsection{TournamentGroup}
Eine TournamentGroup repräsentiert eine Gruppe in der Gruppenphase, also Gruppe A, B, C...

\section{Packet: dbs.project.entity.event}
Diese Entitäten implementieren die abstrakte Klasse MatchEvent.

\subsection{MatchEndEvent}
Ein MatchEndEvent repräsentiert den Schlusspfiff in einem Spiel.

\subsection{PlayerEvent}
Ein PlayerEvent ist ein im Spiel auftretendes Ereignis, das einen Spieler betrifft. Die Klasse ist {\it abstract} und wird von CardEvent, GoalEvent, LineUpEvent und SubstitutionEvent implementiert.

\section{Packet: dbs.project.entity.event.player}
Diese Klassen implementieren die abstrakte Klasse PlayerEvent.

\subsection{CardEvent}
Ein CardEvent repräsentiert die Vergabe einer Karte an einen Spieler.

\subsection{GoalEvent}
Ein GoalEvent repräsentiert ein (Eigen-)Tor.

\subsection{LineUpEvent}
Ein LineUpEvent repräsentiert das Aufstellen eines Spielers in der Startelf.

\subsection{SubstitutionEvent}
Ein SubstitutionEvent repräsentiert das Auswechseln bzw. Einwechseln eines Spielers.

\section{Packet: dbs.project.entity.permission}
Diese Entitäten werden zur Authentifizierung und Rechtevergabe benötigt.

\subsection{Actor}
Ein Actor repräsentiert einen Akteur, der Daten in die Datenbank speichern, oder Daten aus ihr lesen will.

\subsection{Permission}

\subsection{Resource}


\subsection{Role}
%\input{}

\end{document}
